% Essential Formatting
   
\documentclass[12pt]{article}
\usepackage{epsfig,amsmath,amsthm,amssymb}
%\usepackage[ questions]{urmathtest}[2001/05/12]
\usepackage[answersheet]{urmathtest}[2001/05/12]
%\usepackage[answers]{urmathtest}[2001/05/12]

% For use with pdflatex
% \pdfpagewidth\paperwidth
% \pdfpageheight\paperheight

% Basic User Defs
\newcommand{\argmax}{\operatornamewithlimits{argmax}}

\def\ds{\displaystyle}

\newcommand{\ansbox}[1]
{\work{
  \pos\hfill \framebox[#1][l]{ANSWER:\rule[-.3in]{0in}{.7in}}
}{}}

\newcommand{\ansrectangle}
{\work{
  \pos\hfill \framebox[6in][l]{ANSWER:\rule[-.3in]{0in}{.7in}}
}{}}

% Beginning of the Document

\begin{document}
\examtitle{LINEAR REGRESSION MODELS W4315}{INTRO SURVEY}{01/19/2010}
 \begin{center}
  Instructor: Frank Wood (Mathematics 417, Tu-Th, 10:30am-12pm) 
 \end{center}
%%\studentinfo
\instructions{
  %\textbf{Circle your Instructor's Name along with the Lecture Time:}

 

  \begin{itemize}
  \item
    \textbf{Please show all your work.
            Please write your CUID on this page.}
  %\item
   % \textbf{Please put your \underline{simplified}
   %         final answers in the spaces provided.}
  \end{itemize}
}
\finishfirstpage

% Problems Start Here % ----------------------------------------------------- %


\problem{0}
{
\abcs
\item
What is your field of study?  
\item 
What is your intended profession?  
\item
Why are you taking this course?
\item
What regression applications do you have in mind?
\item
What would you really like to learn in this course?
\endabcs
}
{
\vfill
\vspace{3cm}
  %\answer
}
{
\begin{eqnarray}
 \frac{d}{dx})-x^2 + ln(x) ) &=& 0 \\
 -2x+\frac{1}{x} &=& 0 \\
  2x^2 &=& 1 \\
  x &=& \sqrt{\frac{1}{2}} \\
\end{eqnarray}
}

\problem{0}
{
Find the maximum of the function $x^2+ln(x)$, i.e. find  \[\argmax_{x}  \;\;-x^2+ln(x)\]  Simplify your answer.
}
{
\vspace{3cm}

\vfill
  \answer
}
{
\begin{eqnarray}
 \frac{d}{dx})-x^2 + ln(x) -a) &=& 0 \\
 -2x+\frac{1}{x} &=& 0 \\
  2x^2 &=& 1 \\
  x &=& \frac{1}{2} \\
\end{eqnarray}
}

\problem{0}
{
\abcs
\item
Given an algebraic solution to the following matrix equation
\[ {\bf A} {\bf x} - {\bf b} = {\bf 0}\]
\item
What conditions does your answer put on the matrix ${\bf A}$?
\endabcs
}
{
\vspace{3cm}
\vfill
  \answer
}
{
${\bf x} = {\bf A}^{-1}{\bf b}$, ${\bf A}$, ${\bf A}$ must be square and invertible or ${\bf x} = {\bf A}^\dagger{\bf b}$ with fewer restrictions.
}

\problem{0}
{
\abcs
\item
Maximize the following quadratic form w.r.t.~${\bf x}$ ($ \boldsymbol{\mu}$ is also a column vector, ${\bf A}$ and $\bf x$ are as in the previous problem)
\[ ({\bf x - \boldsymbol{\mu}})^T{\bf A}( {\bf x -  \boldsymbol{\mu}} )\]
\endabcs
}
{
\vspace{3cm}
\vfill

  \answer
  \newpage
}
{
$\bf x = \boldsymbol{\mu}$
}

\problem{0}
{
If $X | \lambda \sim \mathrm{Poisson}(\lambda)$ and $\lambda | \alpha, \beta \sim \mathrm{Gamma}(\alpha, \beta)$ how is $X|\alpha, \beta$ distributed?  Hint, remember:
\begin{eqnarray*}
P(a|c) &=& \int P(a|b) P(b|c) db \\
P(X|\lambda) &=& \frac{1}{X!}\lambda^X e^{-\lambda} \\
P(\lambda|\alpha, \beta) &=& \frac{\beta^\alpha}{\Gamma(\alpha)}\lambda^{\alpha-1} e^{-\beta\lambda} 
\end{eqnarray*}
Don't simplify the $\Gamma()$'s in the final result.
}
{
\vspace{3.5in}
\vfill
\newpage
  %\answer

}
{

This is the solution.  
}

\problem{0}
{

Let $\theta_i \sim \mathrm{Exp}(\beta), \, 1 \leq i \leq n$ be samples from an
exponential distribution.  Remember that for an exponential distribution
$P(\theta) = \frac{1}{\beta} e^{-\frac{\theta}{\beta}}$ for $\theta > 0$.  Also
remember that for the exponential distribution $E(\Theta) = \beta$ and $V(\theta) = \beta^2$ where $E()$ stands for
expectation and $V()$ for variance.
\abcs
\item
Derive the maximum likelihood estimator $\hat \beta$ for $\beta$ given observations $\{\theta_i\}_{i=1}^n$.
\item
Is this estimator biased or unbiased?  Show work.  Reminder: the definition of bias is $B(\hat \beta) = \beta - E(\hat \beta)$.
\item
Derive the sample variance of the estimator, $V(\hat \beta)$.  Remember $V(aX) = a^2V(X)$.
\item
Given a set of samples $\{\theta_i\}_{i=1}^n$ as above, illustrate the large
sample (for instance, $N > 30$) symmetric confidence interval for $\hat \beta$.  Remember that by the central limit theorem $\frac{\hat \beta - \beta}{\sqrt{V(\hat \beta)}} \sim N(0,1)$ where $N(0,1)$ stands for a standard normal distribution with mean $0$ and variance $1$.  Also, for a sufficiently large $n$ remember that $\frac{\hat \beta - \beta}{\sqrt{V(\hat \beta)}} \approx \frac{\hat \beta - \beta}{S/\sqrt{n}}$ where $S$ is the sample population standard deviation.  The sample variance is
given by $S^2 = \frac{1}{n-1} \sum_i (\theta_i - \hat \beta)^2$ but
this expansion is unnecessary for the purposes of this question and is provided only for familiarity.  What
you should derive is $c$ for a confidence interval 
\[ \hat \beta - c \leq \beta \leq \hat \beta + c \] 
in terms of $S$, $n$, and $k$ where $k$ value of the standard normal inverse cumulative distribution for some $1-\alpha/2$ level of confidence.  Also show a representative plot of the sampling distribution of the estimator.
\endabcs
}
{
\vspace{7in}
\vfill
  %\answer
}
{

This is the solution.  
}

% Problems End Here % ------------------------------------------------------- %

\problemsdone
\end{document}

% End of the Document
