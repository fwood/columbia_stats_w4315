\documentclass{beamer}
\usepackage{graphicx}
\usepackage{verbatim}
\usepackage{amsmath}
\usepackage{amsfonts}
\usepackage{setspace}
% \usepackage{beamerthemesplit} // Activate for custom appearance

\title{Inference in Normal Regression Model}
\author{Dr. Frank Wood}

\date{}

\DeclareMathOperator*{\Ave}{\mathbb{E}}
\DeclareMathOperator*{\Var}{Var}

\newcommand\independent{\protect\mathpalette{\protect\independenT}{\perp}}
\def\independenT#1#2{\mathrel{\rlap{$#1#2$}\mkern2mu{#1#2}}}
\newcommand{\argmax}{\operatornamewithlimits{argmax}}


\begin{document}

\frame{\titlepage}



\frame[t] {
 \frametitle{Remember}
\begin{itemize}
\item We know that the point estimator of $b_1$ is 
$$b_1=  \frac{\sum(X_i-\bar X)(Y_i - \bar Y)}{\sum(X_i - \bar X)^2}$$
\item Last class we derived the sampling distribution of $b_1$, it being $N(\beta_1,\Var{b_1})$(when $\sigma^2$ known) with
$$\Var{b_1}  = \frac{\sigma^2}{\sum(X_i - \bar X)^2}$$
\item And we suggested that an estimate of $\Var{b_1} $ could be arrived at by substituting the MSE for $\sigma^2$ when $\sigma^2$ is unknown.
$$s^2\{b_1\} = \frac{MSE}{\sum(X_i - \bar X)^2} = \frac{\frac{SSE}{n-2}}{\sum(X_i - \bar X)^2}$$
\end{itemize}
}

\frame[t] {
 \frametitle{Sampling Distribution of $(b_1 - \beta_1)/\EstStd{b_1}$}
\begin{itemize}
\item Since $b_1$ is normally distribute, $(b_1 - \beta_1)/\sigma\{b_1\}$ is a standard normal variable $N(0,1)$
\item We don't know $\Var{b_1} $ so it must be estimated from data.  We have already denoted it's estimate
$s^2\{b_1\}$
\item Using this estimate we showed that
$$\frac{b_1-\beta_1}{s\{b_1\}} \sim t(n-2)$$
where
$$s\{b_1\} = \sqrt{ s^2\{b_1\}}$$
\end{itemize}
It is from this fact that our confidence intervals and tests will derive.
}

\frame[t] {
 \frametitle{Confidence Intervals and Hypothesis Tests}
Now that we know the sampling distribution of $b_1$ (t with n-2
degrees of freedom) we can construct confidence intervals and
hypothesis tests easily }


\frame[t] {%%%change pic%%%
 \frametitle{Confidence Interval for $\beta_1$}
Since the ``studentized'' statistic follows a t distribution we can
make the following probability statement
$$P(t(\alpha/2; n-2)  \leq \frac{b_1-\beta_1}{s\{b_1\}}  \leq t(1-\alpha/2; n-2) ) = 1- \alpha$$
\begin{figure}
  \includegraphics[height=30mm]{1.png}
  \includegraphics[height=30mm]{2.png}
  \includegraphics[height=30mm]{3.png}
\end{figure}
matlab: tpdf, tcdf, tinv
}

\frame[t] {
 \frametitle{Remember}
\begin{itemize}
\item Density: $f(y) = \frac{dF(y)}{dy}$
\item Distribution (CDF): $F(y) = P(Y \leq y) = \int_{-\infty}^yf(t)dt$
\item Inverse CDF: $F^{-1}(p) = y \;\; \mbox{s.t.} \;\; \int_{-\infty}^yf(t)dt = p$
\end{itemize}
}

\frame[t] {
 \frametitle{Book tables and Matlab commands}
Appendix B (or elsewhere in other books), a table of percentiles of the $t$ distribution is given.  In this table one number appears for each of a number of degrees of freedom $\nu$ and a parameter, call it $A$. \newline

Each entry is some value of $t(A; \nu$) where $P\{t(\nu) \leq t(A; \nu)\} = A$ \newline

In words $t(A; \nu)$ is the point on the horizontal axis of the Student-$t$ distribution where $A$ percent of the mass under the curve is located to the left.  This is precisely the quantity returned by $tinv(A,\nu)$ in Matlab.\newline

How can this be used to produce a confidence interval?

}

\frame[t] {
 \frametitle{Interval arriving from picking $\alpha$}
\begin{itemize}
\item Note that by symmetry $$t(\alpha/2; n-2) = -t(1-\alpha/2; n-2)$$
\item Remember
$$P(t(\alpha/2; n-2)  \leq \frac{b_1-\beta_1}{s\{b_1\}}  \leq t(1-\alpha/2; n-2) ) = 1- \alpha$$
\item Rearranging terms and using this symmetry we have
{\small
\hspace{-1cm}\begin{eqnarray*}P(b_1 - t(1-\alpha/2; n-2 )s\{b_1\} \leq \beta_1  \leq b_1 +
t(1-\alpha/2; n-2 ) s\{b_1\}) \\= 1- \alpha\end{eqnarray*}}
\item And now we can use a table to look up and produce confidence intervals
\end{itemize}
}

\frame[t] {
 \frametitle{Using tables for Computing Intervals }
\begin{itemize}
\item The tables in the book (table B.2 in the appendix) for $t(1-\alpha/2;\nu)$ where
$P\{t(\nu)\leq t(1-\alpha/2;\nu)\} = A$
\item Provides the inverse CDF of the t-distribution
\item This can be arrived at computationally as well\\
Matlab: $tinv(1-\alpha/2, \nu)$
\end{itemize}
}

\frame[t] {
 \frametitle{$1-\alpha$ confidence limits for $\beta_1$}
\begin{itemize}
\item The $1-\alpha$ confidence limits for $\beta_1$ are
$$b_1 \pm  t(1-\alpha/2; n-2 )s\{b_1\}$$
\item Note that this quantity can be used to calculate confidence intervals given n and $\alpha$.
\begin{itemize}
\item Fixing $\alpha$ can guide the choice of sample size if a particular confidence interval is desired
\item Give a sample size, vice versa.
\end{itemize}
\item Also useful for hypothesis testing
\end{itemize}
}

\frame[t] {
 \frametitle{Tests Concerning $\beta_1$}
\begin{itemize}
\item Example 1
\begin{itemize}
\item Two-sided test
\begin{itemize}
\item $H_0 : \beta_1 = 0$
\item $H_a : \beta_1 \neq 0$
\item Test statistic
$$t^* = \frac{b_1-0}{s\{b_1\} }$$
\end{itemize}
\end{itemize}
\end{itemize}
}

\frame[t] {
 \frametitle{Tests Concerning $\beta_1$}
\begin{itemize}
\item We have an estimate of the sampling distribution of $b_1$ from the data.
\item If the null hypothesis holds then the $b_1$ estimate coming from the data
should be within the $95\%$ confidence interval of the sampling
distribution centered at 0 (in this case)
$$t^* = \frac{b_1-0}{s\{b_1\} }$$
\item Variability in $b_1$ is assumed to arise from sampling noise.
\end{itemize}
}

\frame[t] {
 \frametitle{Decision rules}
\begin{eqnarray*}
\mathrm{if }\,  |t^*| &\leq& t(1-\alpha/2;n-2)\mathrm{, conclude } \, H_0\\
\mathrm{if }\,  |t^*| &>& t(1-\alpha/2;n-2)\mathrm{, conclude }\,
H_\alpha
\end{eqnarray*}
Absolute values make the test two-sided }

\frame[t] {%%%change pic%%%
 \frametitle{Intuition}
\begin{figure}
  \includegraphics[height=60mm]{intuition.png}
\end{figure}
p-value is value of $\alpha$ that moves the green line to the blue
line }

\frame[t] {
 \frametitle{Calculating the p-value}
\begin{itemize}
\item The p-value, or attained significance level, is the smallest level of significance $\alpha$
for which the observed data indicate that the null hypothesis should be rejected.
\item This can be looked up using the CDF of the test statistic.
\item In Matlab\\
Two-sided p-value\\
$2*(1-tcdf(|t^*|,\nu))$

\end{itemize}
}

\frame[t] {
 \frametitle{Inferences Concerning $\beta_0$}
\begin{itemize}
\item Largely, inference procedures regarding $\beta_0$ can be performed in the same way as those
for $\beta_1$
\item Remember the point estimator $b_0$ for $\beta_0$
$$b_0 = \bar Y - b_1 \bar X$$
\end{itemize}
}

\frame[t] {
 \frametitle{Sampling distribution of $b_0$}
\begin{itemize}
\item The sampling distribution of $b_0$ refers to the different values of $b_0$
that would be obtained with repeated sampling when the levels of the
predictor variable X are held constant from sample to sample.
\item For the normal regression model the sampling distribution of $b_0$ is normal

\end{itemize}
}

\frame[t] {
 \frametitle{Sampling distribution of $b_0$}
\begin{itemize}
\item When error variance is known
$$\Ave{b_0} = \beta_0$$
$$\sigma^2\{b_0\} = \sigma^2 (\frac{1}{n} + \frac{\bar X^2}{\sum(X_i - \bar X)^2})$$
\item When error variance is unknown
$$s^2\{b_0\}= MSE (\frac{1}{n} + \frac{\bar X^2}{\sum(X_i - \bar
X)^2})$$
\end{itemize}
}

\frame[t] {
 \frametitle{Confidence interval for $\beta_0$}
The $1-\alpha$ confidence limits for $\beta_0$ are obtained in the
same manner as those for $\beta_1$
$$b_0 \pm  t(1-\alpha/2; n-2 )s\{b_0\}$$
}

\frame[t] {
 \frametitle{Considerations on Inferences on $\beta_0$ and $\beta_1$}
\begin{itemize}
\item Effects of departures from normality
\begin{itemize}
\item The estimators of $\beta_0$ and $\beta_1$ have the property of
asymptotic normality - their distributions approach normality as the
sample size increases (under general conditions)
\end{itemize}


\item Spacing of the X levels
\begin{itemize}
\item The variances of $b_0$ and $b_1$ (for a given n and $\sigma^2$)
depend strongly on the spacing of X

\end{itemize}


\end{itemize}
}

\frame[t] {
 \frametitle{Sampling distribution of point estimator of mean response}
\begin{itemize}
\item Let $X_h$ be the level of X for which we would like an estimate of the mean
response\\
Needs to be one of the observed X's
\item The mean response when $X=X_h$ is denoted by $\Ave{Y_h}$
\item The point estimator of $\Ave{Y_h}$ is
$$\hat Y_h = b_0 + b_1 X_h$$
We are interested in the sampling distribution of this quantity
\end{itemize}
}

\frame[t] {
 \frametitle{Sampling Distribution of $\hat Y_h$}
\begin{itemize}
\item We have
$$\hat Y_h = b_0 + b_1 X_h$$
\item Since this quantity is itself a linear combination of the $Y_i's$ it's sampling distribution is itself normal.
\item The mean of the sampling distribution is $$E\{\hat Y_h\} = E\{b_0\} + E\{b_1\} X_h = \beta_0 + \beta_1X_h$$
Biased or unbiased?
\end{itemize}
}

\frame[t] {
 \frametitle{Sampling Distribution of $\hat Y_h$}
\begin{itemize}
\item To derive the sampling distribution variance of the mean response we first show that
$b_1$ and $(1/n)\sum Y_i$ are uncorrelated and, hence, for the
normal error regression model independent
\item We start with the definitions
$$\bar Y = \sum (\frac{1}{n}) Y_i$$
\begin{eqnarray*}
b_1 &=& \sum k_i Y_i, \, k_i = \frac{(X_i - \bar X)}{\sum(X_i-\bar
X)^2}
\end{eqnarray*}

\end{itemize}
}

\frame[t] {
 \frametitle{Sampling Distribution of $\hat Y_h$}
\begin{itemize}
\item We want to show that mean response and the estimate $b_1$ are uncorrelated
$$Cov(\bar Y, b_1) = \sigma^2\{\bar Y, b_1\} = 0$$
\item To do this we need the following result (A.32)
$$\sigma^2\{\sum_{i=1}^n a_i Y_i, \sum_{i=1}^n c_i Y_i\} = \sum_{i=1}^n a_i c_i \sigma^2\{Y_i\}$$
when the $Y_i$ are independent
\end{itemize}
}


\frame[t] {
 \frametitle{Sampling Distribution of $\hat Y_h$}
Using this fact we have
\begin{eqnarray*}
\sigma^2\{\sum_{i=1}^n \frac{1}{n} Y_i, \sum_{i=1}^n k_i Y_i\} &=& \sum_{i=1}^n \frac{1}{n} k_i \sigma^2\{Y_i\} \\
&=&  \sum_{i=1}^n \frac{1}{n} k_i \sigma^2 \\
&=& \frac{\sigma^2 }{n}  \sum_{i=1}^n k_i \\
&=& 0
\end{eqnarray*}
So the $\bar{Y}$ and $b_1$ are uncorrelated }

\frame[t] {
 \frametitle{Sampling Distribution of $\hat Y_h$}
\begin{itemize}
\item This means that we can write down the variance
$$\sigma^2\{\hat Y_h\} = \sigma^2\{\bar Y + b_1(X_h - \bar X)\}$$
alternative and equivalent form of regression function
\item But we know that the mean of Y and $b_1$ are uncorrelated so
$$\sigma^2\{\hat Y_h\} = \sigma^2\{\bar Y\} + \sigma^2\{b_1\}(X_h - \bar X)^2$$
\end{itemize}
}

\frame[t] {
 \frametitle{Sampling Distribution of $\hat Y_h$}
\begin{itemize}
\item We know (from last lecture)
\begin{eqnarray*}
\sigma^2\{b_1\} &=&  \frac{\sigma^2}{\sum(X_i - \bar X)^2} \\
s^2\{b_1\} &=&  \frac{MSE}{\sum(X_i - \bar X)^2}
\end{eqnarray*}
\item And we can find
$$\sigma^2\{\bar Y\} = \frac{1}{n^2} \sum  \sigma^2\{Y_i\} =  \frac{n \sigma^2 }{n^2} =   \frac{ \sigma^2 }{n}$$
\end{itemize}
}

\frame[t] {
 \frametitle{Sampling Distribution of $\hat Y_h$}
\begin{itemize}
\item So, plugging in, we get
$$\sigma^2\{\hat Y_h\} = \frac{\sigma^2}{n}+ \frac{\sigma^2}{\sum(X_i - \bar X)^2}(X_h - \bar X)^2$$
\item Or
$$\sigma^2\{\hat Y_h\} = \sigma^2\left(\frac{1}{n}+ \frac{(X_h - \bar X)^2 }{\sum(X_i - \bar X)^2}\right)$$
\end{itemize}
}

\frame[t] {
 \frametitle{Sampling Distribution of $\hat Y_h$}
Since we often won't know $\sigma^2$ we can, as usual, plug in $S^2
= SSE/(n-2)$, our estimate for it to get our estimate of this
sampling distribution variance $$s^2\{\hat Y_h\} =
S^2\left(\frac{1}{n}+ \frac{(X_h - \bar X)^2 }{\sum(X_i - \bar
X)^2}\right)$$ }

\frame[t] {
 \frametitle{No surprise$\ldots$}
\begin{itemize}
\item The sampling distribution of our point estimator for the output is distributed as a t-distribution with two degrees of freedom
$$\frac{\hat Y_h - E\{Y_h\}}{s\{\hat Y_h\}} \sim t(n-2)$$
\item This means that we can construct confidence intervals in the same manner as before.
\end{itemize}
}

\frame[t] {
 \frametitle{Confidence Intervals for $\Ave{Y_h}$}
\begin{itemize}
\item The $1-\alpha$ confidence intervals for $\Ave{Y_h}$ are
$$\hat Y_h \pm t(1-\alpha/2; n-2)s\{\hat Y_h\}$$
\item From this hypothesis tests can be constructed as usual.

\end{itemize}
}

\frame[t] {
 \frametitle{Comments}
\begin{itemize}
\item The variance of the estimator $\hat {Y}_h$ is smallest near the mean of X.
Designing studies such that the mean of X is near $X_h$ will improve
inference precision
\item When $X_h$ is zero the variance of the estimator $\hat{Y}_h$ reduces to the variance of the estimator $b_0$ for
$\beta_0$
\end{itemize}
}

\frame[t] {
 \frametitle{Prediction interval  for {\em new} input $X_{h}$}
\begin{itemize}
\item Roughly the same idea as for $\Ave{Y_h}$ where $X_h$ is a known input point included in the estimation of $b_1, b_0,$ and $s^2$
\item If all regression parameters are known then the $1-\alpha$ prediction interval for a new observation $Y_h$ is
$$\Ave{Y_h} \pm z(1-\alpha/2)\sigma$$
\end{itemize}
}

\frame[t] {
 \frametitle{Prediction interval  for {\em new} input $X_{h}$}
\begin{itemize}
\item If the regression parameters are unknown the $1-\alpha$ prediction interval for a new observation $Y_{h(new)}$ is given by the following theorem
\[ \frac{Y_{h(new)} - \hat Y_{h} }{\EstStd{\mathrm{pred}}} \sim t(n-2)\]
for the normal error regression model. $\EstStd{\mathrm{pred}}$ to be defined shortly. \newline

It follows directly that the $1-\alpha$ prediction limits for $Y_{h(new)}$ are 
$$\hat Y_{h} \pm t(1-\alpha/2; n-2){s\{pred\}}$$
\item This is very nearly the same as prediction for a known value of X but includes a correction for
the fact that there is additional variability arising from the fact
that the new input location was not used in the original estimates
of $b_1$, $b_0$, and $s^2$
\end{itemize}
}

\frame[t] {
 \frametitle{Prediction interval  for {\em new} input $X_{h}$}

Because $Y_{h(new)}$ is independent of $\hat Y_{h}$ we can directly write 

\[\Var{\mathrm{pred}} = \Var{ Y_{h(new)} - \hat{Y}_{h} } = \Var{ Y_{h(new)}} +  \Var{  \hat{Y}_{h}} = \sigma^2 + \Var{\hat{Y}_h}\]

where from before we have that
$$\sigma^2\{\hat Y_h\} = \sigma^2\left(\frac{1}{n}+ \frac{(X_h - \bar X)^2 }{\sum(X_i - \bar X)^2}\right)$$
so 
\[\Var{\mathrm{pred}} = \sigma^2\left[1 + \frac{1}{n}+ \frac{(X_h - \bar X)^2 }{\sum(X_i - \bar X)^2}\right]\]
but as before we don't know $\sigma^2$ so we will replace it...
}

\frame[t] {
 \frametitle{Prediction interval for {\em new} input $X_{h}$}
The value of $s^2\{pred\}$ is given by
$${s^2\{pred\}}  = MSE\left[ 1 + \frac{1}{n} + \frac{(X_h-\bar X)^2}{\sum (X_i - \bar X)^2}\right]$$
Note that this quantity is {\em slightly} larger than $s^2\{\hat Y_h\}$. \newline

It has two components
\begin{itemize}
\item The variance of the distribution of $y$ at $X = X_h,$ namely $\sigma^2$
\item The variance of the sampling distribution of $\hat{Y}_h,$ namely $\EstVar{\hat{Y}_h}$.
 \end{itemize}

}

\frame[t] {
 \frametitle{Summary}
After this lecture you should be able to confidently do estimation, prediction, and hypothesis testing about the slope, intercept, and predicted values at any input point, old or new in the normal error linear regression setting.
}



%
%\frame[t] {
% \frametitle{Where does this come from?}
%\begin{itemize}
%\item We needed to rely upon (but didn't derive) the following theorem\\
%\bigskip
%For the normal error regression model
%$$\frac{SSE}{\sigma^2} = \frac{\sum (Y_i - \hat Y_i)^2}{\sigma^2} \sim \chi^2(n-2)$$
%and is independent of $b_0$ and $b_1$.\\
%\bigskip
%\item Here there are two linear constraints
%\begin{eqnarray*}
%b_1 &=&  \frac{\sum(X_i-\bar X)(Y_i - \bar Y)}{\sum(X_i - \bar X)^2} = \sum_i k_i Y_i,\;\;\; k_i = \frac{X_i - \bar X}{\sum_i (X_i - \bar X)^2}\\
%b_0 &=& \bar Y - b_1 \bar X
%\end{eqnarray*}
%
% imposed by the regression
%parameter estimation that each reduce the number of degrees of
%freedom by one (total two).
%\end{itemize}
%}
%
%\frame[t] {
% \frametitle{Reminder: normal (non-regression) estimation}
%\begin{itemize}
%\item Intuitively the regression result from the previous slide follows the standard result for the sum of squared standard normal random variables.  First, with $\sigma$ and $\mu$ known
%$$\sum_{i=1}^n Z_i^2 = \sum_{i=1}^n \left( \frac{Y_i-\mu}{\sigma} \right)^2 \sim \chi^2(n)$$
%and then with $\mu$ unknown
%\begin{eqnarray*}
%S^2 &=& \frac{1}{n-1} \sum_{i=1}^n (Y_i-\bar Y)^2 \\
%\frac{(n-1)S^2}{\sigma^2} &=& \sum_{i=1}^n \left( \frac{Y_i-\bar Y}{\sigma} \right)^2 \sim \chi^2(n-1)
%\end{eqnarray*}
%and $\bar Y$ and $S^2$ are independent. \footnote{Wackerly, pg. 335}
%\end{itemize}
%}
%
%\frame[t] {
% \frametitle{Reminder: normal (non-regression) estimation cont.}
%\begin{itemize}
%\item With both  $\mu$ and $\sigma$ unknown then
%$$ \sqrt{n}\left(\frac{\bar Y - \mu}{S}\right) \sim t(n-1) $$
%\end{itemize}
%
%because
%$$ T = \frac{Z}{\sqrt{W/\nu}} = \frac{ \sqrt{n}(\bar Y - \mu)/\sigma}{\sqrt{[(n-1)S^2/\sigma^2]/(n-1)}} = \sqrt{n}\left(\frac{\bar Y - \mu}{S}\right) $$
%\bigskip
%Here the numerator follows from 
%$$ Z = \frac{\bar Y - \mu_{\bar Y}}{\sigma_{\bar Y}} = \frac{\bar Y - \mu}{\sqrt{\frac{\sigma^2}{n}}}$$
%
%}
%
%\frame[t] {
% \frametitle{Another useful fact : Student-t distribution}
%Let $Z$ and $\chi^2(\nu)$ be independent random variables (standard
%normal and $\chi^2$ respectively).  We then define a t random
%variable as follows:
%$$t(\nu) = \frac{Z}{\sqrt{\frac{\chi^2(\nu)}{\nu}}}$$
%This version of the t distribution has one parameter, the degrees of
%freedom $\nu$
%
%}
%
%\frame[t] {
% \frametitle{Distribution of the studentized statistic}
%To derive the distribution of the statistic $\frac{b_1 - \beta_1}{s\{b_1\}} $ first we do the
%following rewrite
%$$\frac{b_1 - \beta_1}{s\{b_1\}} = \frac{\frac{b_1 - \beta_1}{{ \sigma\{b_1\}}}}{\frac{s\{b_1\}}{ \sigma\{b_1\} }}$$
%where
%$$\frac{s\{b_1\}}{ \sigma\{b_1\} }= \sqrt{\frac{s^2\{b_1\}}{ \sigma^2\{b_1\} }}$$}
%
%\frame[t] {
% \frametitle{Studentized statistic cont.}
%And note the following
%$$\frac{s^2\{b_1\}}{ \sigma^2\{b_1\} }= \frac{\frac{MSE}{\sum(X_i-\bar X)^2}}{\frac{\sigma^2}{\sum(X_i-\bar X)^2}} = \frac{MSE}{\sigma^2} = \frac{SSE}{\sigma^2(n-2)}$$
%where we know (by the given theorem) the distribution of the last
%term is $\chi^2$ and indep. of $b_1$ and $b_0$
%$$\frac{SSE}{\sigma^2(n-2)} \sim \frac{\chi^2(n-2)}{n-2}$$}
%
%\frame[t] {
% \frametitle{Studentized statistic final}
%But by the given definition of the t distribution we have our result
%$$\frac{b_1-\beta_1}{s\{b_1\}} \sim t(n-2)$$
%because putting everything together we can see that $$\frac{b_1 -
%\beta_1}{s\{b_1\}} \sim
%\frac{z}{\sqrt{\frac{\chi^2(n-2)}{n-2}}}$$ }


\end{document}
