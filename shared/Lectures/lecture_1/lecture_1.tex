\documentclass{beamer}
\usepackage{graphicx}
\usepackage{verbatim}
\usepackage{amsmath}
\usepackage{amsfonts}
\usepackage{setspace}
% \usepackage{beamerthemesplit} // Activate for custom appearance

\title{Linear Regression Models\\W4315}
\author{Instructor: Dr. Frank Wood}

\date{Required Text: Applied Linear Regression\\
Authors: Kutner, Nachtsheim, Neter}



\DeclareMathOperator*{\Ave}{\mathbb{E}}
\DeclareMathOperator*{\Var}{Var}

\begin{document}

\frame{\titlepage}


\frame[t] {
 \frametitle{Not Registered Yet?}
Fill out the form at\\
\href{http://tinyurl.com/3jv6mdl}{http://tinyurl.com/3jv6mdl}
\bigskip

Additional books we will draw material from in this course:
\begin{itemize}
\item Pattern Recognition and Machine Learning, by Christopher M. Bishop. Springer, 2006.
\item Bayesian Data Analysis, Second Edition, by Andrew Gelman, John B. Carlin, Hal S. Stern, and Donald B. Rubin, Chapman \& Hall/CRC Texts in Statistical Science
\end{itemize}




}

\frame[t] {
 \frametitle{Course Description}
Theory and practice of regression analysis, Simple and multiple
regression, including testing, estimation, and confidence
procedures, modeling, regression diagnostics and plots, polynomial
regression, colinearity and confounding, model selection, geometry
of least squares. Extensive use of the computer to analyze data.
\newline

Course website
or \href{http://www.stat.columbia.edu/$\sim$fwood/w4315/}{\underline{http://www.stat.columbia.edu/$\sim$fwood/w4315/}}

}

\frame[t] {
 \frametitle{Philosophy and Style}
\begin{itemize}
\item Easy first half.
\item Very hard second half.
\item Frequent, long digressions from the required book.
\item Understanding == proof (derivation) {\em plus} implementation.
\item Practice makes perfect.
\item Frequentist {\em and} Bayesian perspectives taught.
\end{itemize}

If you are looking for a pure applied, pure frequentist treatment of regression as a diagnostic tool and/or you've never programmed before, seek another section.
\newline 

Available sections
\begin{itemize}
\item MW 10:35am-11:50am, (Zheng)
\item MW 6:10pm-7:25pm (Stodden)
\item F 10:00am-12:30pm (Lindquist)
\end{itemize}

}

\frame[t] {
 \frametitle{Goals}
\begin{itemize}
\item Deep theoretical understanding
\begin{itemize}
\item Book provides only recipes
\item Much detail missing
\item Can always look up recipes in future
\end{itemize}
\item Ability to implement/code {\em all} regression functionality
\begin{itemize}
\item Different levels of understanding
\item Not enough to simply be able to apply formula and use pre-built regression software
\end{itemize}

\end{itemize}

}


\frame[t] {
 \frametitle{About me}
\begin{itemize}
\item Computer Science PhD, 2007, Brown University
\item Postdoc in Machine Learning, Gatsby Unit, University College London
\item Sports gambling consulting.
\item Former entrepreneur.
\end{itemize}
My research
\begin{itemize}
\item Inference for nonparametric Bayesian models.
\item Compression.
\item Natural language data  modeling.
\end{itemize}

My website: \href{http://www.stat.columbia.edu/~fwood}{http://www.stat.columbia.edu/$\sim$fwood}


}


\frame[t] {
 \frametitle{Course Outline}
The first half of the first half is a formal, theoretical review of single variable regression and its classical, frequentist treatment.

 \begin{itemize}
 \item Roughly 1 chapter per week
 \item 3-5 weeks, linear regression
 \begin{itemize}
 \item Least squares
 \item Maximum likelihood, normal model
 \item Tests / inferences
 \item ANOVA
 \item Diagnostics
 \item Remedial Measures
 \item Linear algebra review
 \item Matrix approach to linear regression
\end{itemize}
\end{itemize}
}


\frame[t] {

The second half of the first half covers multiple regression and the various topics that arise from including multiple predictor variables into models.

 \frametitle{Course Outline Continued}
 \begin{itemize}
 \item 3-4 weeks multiple regression
 \begin{itemize}
 \item Multiple predictor variables
 \item Diagnostics
 \item Tests
 \end{itemize}
 \end{itemize}
 Midterm

}


\frame[t] {

The remainder of the course will deviate from the book and may be ordered differently than what is shown here.  In general we will retain a focus on models that are linear in the parameters, but will look at nonlinear models and Bayesian treatments of linear models.

 \frametitle{Course Outline Continued}
 \begin{itemize}
 \item 3-4 weeks on Bayesian regression
 \begin{itemize}
 \item MCMC
 \item Bayesian linear regression
 \item Gaussian process regression
 \item Projects
 \end{itemize}
 \end{itemize}
  \begin{itemize}
 \item 3-4 weeks on generalized regression
 \begin{itemize}
 \item Polynomial regression
 \item Logistic regression
 \item Neural networks
 \item Generalized linear models
 \end{itemize}
 \end{itemize}

}


\frame[t] {
 \frametitle{Requirements}
 \begin{itemize}
 \item Calculus
 \begin{itemize}
 \item Derivatives, gradients, convexity
 \end{itemize}
 \end{itemize}
 \begin{itemize}
 \item Linear algebra
 \begin{itemize}
 \item Matrix notation, inversion, eigenvectors, eigenvalues, rank, quadratic forms
 \end{itemize}
 \end{itemize}
 \begin{itemize}
 \item Probability
 \begin{itemize}
 \item Random variables
 \item Bayes Rule
 \end{itemize}
 \end{itemize}
 \begin{itemize}
 \item Statistics
 \begin{itemize}
 \item Expectation, variance
 \item Estimation
 \item Bias/Variance
 \item Basic probability distributions
 \end{itemize}
 \end{itemize}
 \begin{itemize}
 \item Programming
 \end{itemize}

}


\frame[t] {
 \frametitle{Projects (homework and final)}
 \begin{itemize}
 \item Software

 \begin{itemize}
 \item For homework -- Matlab.
 \end{itemize}
 \end{itemize}

}

\frame[t] {
 \frametitle{Grading}
 \begin{itemize}
 \item Bi-weekly homework $(25\%)$
 \begin{itemize}
 \item Due every other week
 \begin{itemize}
 \item no late homework accepted
 \end{itemize}
 \item None skipped
 \end{itemize}
  \end{itemize}
 \begin{itemize}
 \item Participation (up a half grade if I know you by the end, down a half grade if not)
 \item Midterm examination $(25\%)$
 \item Final project $(25\%)$
 \item Quizzes $(25\%)$
 \item Curve
 \end{itemize}

}

\frame[t] {
 \frametitle{Office Hours / Website}
 \begin{itemize}
 \item $http://www.stat.columbia.edu/\sim fwood$
 \item Office hours : TBA
 \item Office : Room 1017
 \item TA : Ran He
 \begin{itemize}
 \item TA office hours TBD
 \item Email: ran@stat.columbia.edu
 \end{itemize}
 \end{itemize}
}

\frame[t] {
 \frametitle{Why regression?}
 \begin{itemize}
 \item Want to model a functional relationship between an ``predictor'' (input, independent variable, single or multiple, etc.)
      and a ``response variable'' (output, dependent variable, potentially many simultaneous, etc.)
 \begin{itemize}
 \item Examples?
 \end{itemize}
 \end{itemize}
 \begin{itemize}
 \item But real world is noisy, no $f = ma$
 \begin{itemize}
 \item Observation noise
 \item Process noise
 \end{itemize}
 \item Two distinct goals
 \begin{itemize}
 \item Tests about natura of relationship between predictor variables and response variables.
 \begin{itemize}
\item Positive 
\item Negative
\item No effect
\end{itemize}

 \item Prediction
 \end{itemize}
 \end{itemize}
}

\frame[t] {
 \frametitle{History}
 \begin{itemize}
 \item Sir Francis Galton, $19^{th}$ century
 \begin{itemize}
 \item Studied the relation between heights of parents and children and noted that the children ``regressed'' to the population mean
 \end{itemize}
 \end{itemize}
 \begin{itemize}
 \item ``Regression'' stuck as the term to describe statistical relations between variables
 \end{itemize}
}
\frame[t] {
 \frametitle{Example Applications}
 \begin{itemize}
 \item Epidemiology
 \begin{itemize}
 \item Relating lifespan to obesity, smoking habits, and/or other patient {\em features}.
 \end{itemize}
 \end{itemize}
 \begin{itemize}
 \item Science and engineering
 \begin{itemize}
 \item Relating physical inputs to physical outputs in complex systems
 \end{itemize}
 \end{itemize}
 \begin{itemize}
 \item Grander

 \begin{center}
 \begin{figure}
  \hspace{-2cm}
  \includegraphics[height=20mm]{brain.png}
\end{figure}
\end{center}
 \end{itemize}
}

\frame[t] {
 \frametitle{Aims for the course}
\begin{itemize}
\item Given something you would like to predict and some number of covariates
\begin{itemize}
\item What kind of model should you use?
\item Which variables should you include?
\item Which transformations of variables and interaction terms should you use?
\end{itemize}
\item Given a model and some data
\begin{itemize}
\item How do you fit the model to the data?
\item How do you express confidence in the values of the model parameters?
\item How do you regularize the model to avoid over-fitting and other related issues?
\end{itemize}
\item Not be boring
\end{itemize}

}

\end{document}
