\documentclass[serif, professionalfont]{beamer}
\usetheme{Frankfurt}

% \usepackage{fourier}
% \usepackage{eulervm}
\usepackage[T1]{fontenc}
\usepackage{fourier}
\usepackage{color}
\usepackage{amsmath,amsthm,amssymb}
% \usepackage{beamerthemesplit} // Activate for custom appearance

\title{Count Response, Categorical Response and Generalized Linear Model Framework}
\author{Wei Wang}
\date{Dec 9, 2010}
% \date{\today}


\begin{document}

\frame{\titlepage}

% \section[Outline]{}
% \frame{\tableofcontents}
% 
% \section{Introduction}
% \subsection{Overview of Topics}
% 
% \section{Bayesian Analysis}
% \subsection{Single Parameter Model}

\frame[t]{
  \frametitle{Outline}
  \begin{itemize}
  \item Count Response
  \vskip 12pt
  \item Categorical Response
  \vskip 12pt
  \item Generalized Linear Models
  \vskip 12pt
  \end{itemize}
}

\frame[t] {
  \frametitle{Count Data}
  \begin{itemize}
  \item Other than 0-1 binary data, {\color{red}count data} is another discrete data type that we deal with everyday.
    \vskip 24pt
  \item number of traffic accidents, number of epidemic incidences etc.
    \vskip 24pt
  \item Again we need to link predictors $X$ and response $y$.
    \vskip 24pt
  \end{itemize}
}


\frame[t] {
  \frametitle{Binomial Regression (Extension of Logistic Regression)}
  \begin{itemize}
  \item If response $y_i$ can be naturally interprested as number of success in $n_i$ Bernoulli experiments, then we can still use logistic model.
    \vskip 24pt
  \item We only need to treat each observation as $n_i$ data points, with $y_i$ 1's and $(n_i-y_i)$ 0's.
  \end{itemize}

}

\frame[t] {
  \frametitle{Example}
  \begin{itemize}
  \item We study the proportion of death penalty that were overturned in 34 states in 2000. For each state, $n_i$ is the total number of death sentences in 2000 and $y_i$ is the number of death penalty cases that were overturned.
    \vskip 24pt
  \item  Very naturally, logistic regression is a good model for this study.
  \end{itemize}

}

\frame[t] {
  \frametitle{Poisson Model}
  \begin{itemize}
  \item But in a lot of cases, a binomial model is not appropriate.
    \vskip 16pt 
  \item 
    No natural explanation of success/failure rate; no natural limit of maximum number of incidences.
    \vskip 12pt
  \item A more flexible way to model count data is {\color{red}Poisson distribution}.
    \[P(N=n)=\exp(-\lambda)\frac{\lambda^n}{n!}\]
    \vskip 12pt
  \item The parameter (mean) of Poisson distribution $\lambda$ is positive, so we need a transformation that maps $x^T
    \beta$ to $[0,+\infty)$.
  \end{itemize}
}


\frame[t] {
  \frametitle{Poisson Regression(Loglinear Model)}
  \begin{itemize}
  \item An exponential transformation is a natural choice. 
    \vskip 12pt
  \item Formally, a {\color{red}Poisson Regression (Loglinear Model)} is given by 
    \begin{eqnarray*}
     P(y_i=k)=\exp(-\exp(X_i^T\beta))\frac{(\exp(X_i^T\beta))^k}{k!}
    \end{eqnarray*}
    \vskip 12pt
  \item Similar to logistic regression, there is no variance component in Poisson Regression. 
    \vskip 12pt
  \item But unlike logistic regression, we will see that absence of variance parameter could cause us trouble in model fitting.
    
  \end{itemize}

}

\frame[t]{
  \frametitle{Example: What cause traffic accidents?}
  \begin{itemize}
  \item
    We study the number of traffic accidents at a group of street intersections. Intersections are indexed by $i$. $y_i$ is the number of traffic accidents in a particular year. For predictors, we have $X_1$, the average speed of vehicles at that intersection, and $X_2$, the indicator of whether the intersection has a traffic signal.
    \vskip 24pt
  \item 
    With standard statistical software, we have the fitted model 
    \[y_i\sim \text{Poisson}(\exp(2.8+0.012X_{i1}-0.20X_{i2}))\]
    \vskip 12pt
  \item At first sight, the signs of the coefficient estimates make sense.
  \end{itemize}

}



\frame[t] {
  \frametitle{Interpretation of the Coefficients}
  \begin{itemize}
  \item The slope of $X_1$ is the expected difference in $y$ (on the log scale) for each additional mile-per-hour increase of the average speed of vehicles. The multiplicative increase is an $e^{0.012}-1=1.012-1=1.2\%$ positive difference in the rate of traffic accidents. 
    \vskip 12pt
  \item The parameter of $X_2$ tells us the expected difference in $y$ (on the log scale) if the intersection has a traffic signal. The multiplicative decrease is an $1-e^{0.20}=18\%$ in the rate of traffic accidents.

  \end{itemize}

}


\frame[t] {
  \frametitle{Offset}
  \begin{itemize}
    
  \item In most application of Poisson Regression, we want to interpret the count relative to some baseline or "exposure".
    \vskip 12pt
  \item
    In the traffic accidents example, it is natural to think that the rate of traffic accidents at one particular intersection should be proportional to the total number of vehicles that passed that intersection, which we denote as $u_i$.
    \vskip 12pt
  \item
    So the Poisson Regression should be expressed as 
    \[y_i\sim \text{Poisson}(u_i\exp(x_i^T\beta))\]
    \vskip 12pt
  \item $\log(u_i)$ is called {\color{red}offset} in Poisson Regression terminology.
  \end{itemize}

}

\frame[t]{
  \frametitle{Fitting the Model}
  \begin{itemize}
  \item In $\texttt{R}$, it looks like this
    \\
    \texttt{glm(accident $\sim$ ave\_speed + traf\_light,family=poisson, offset=log(num\_vehicle))}
  \vskip 24pt
  \pause
\item In $\texttt{Matlab}$, I don't really know how it looks like.
\end{itemize}
}

\frame[t] {
  \frametitle{Overdispersion}
  \begin{itemize}
  \item For Poisson distribution, its variance is equal to its expecation. 
    \vskip 12pt
  \item But with real data, this requirement is often violated. In most of the cases, we have larger variance than expected if Poisson distribution really holds. It is called {\color{red}overdispersion}.
    \vskip 12pt
  \item It almost always happens with count data.
    \vskip 12pt
  \item We can see the tradeoff between the parismony and flexibility of the model.
  \end{itemize}
}

\frame[t]{
  \frametitle{Detecting Overdispersion}
  \begin{itemize}
  \item Under Poisson distribution, the standardized residuals 
    \[z_i=\frac{y_t-\hat{y}_t}{sd\{y_t\}}=\frac{y_t-\hat{y}_t}{\sqrt{\hat{y}_t}}\]
    should be approximately independently distributed and have mean 0 and sd 1.
  \item The sum of square of $z_i$ follows a $\chi^2_{n-p}$ distribution, where $p$ is the number of predictors (including intercept).
    \[\sum_{i=1}^nz_i^2\to \chi^2_{n-p}\]
  \item In the case of overdispersion, all the $z_i$'s are larger than 1 in absolute values and the sum of square should be much larger than $n-p$.   
  \end{itemize}
}

\frame[t]{
  \frametitle{Dealing with Overdispersion}
  \begin{itemize}
    \item As we mentioned before, the problem of overdispersion is caused by the lack of variance parameter in Poisson distribution. 
    \vskip 12pt
    \item So the solution is to loosen the Poisson distribution to Quasi Poisson distribution, which includes an overdispersion parameter $\omega$, which is the ratio of variance and mean of the distribution. 
    \vskip 12pt
    \item \texttt{glm(accident $\sim$ ave\_speed + traf\_light, family=quasipoisson, offset=log(num\_vehicle))}
    \vskip 12pt
    \item Overdispersed-Poisson is a class of distribution, and we commonly use Negative-Binomial distribution.
  \end{itemize}
}

\frame[t] {
  \frametitle{Categorical Response}
  \begin{itemize}
  \item Categorical Response can be either ordered (ordinal) or unordered (nominal).
  \vskip 12pt
  \item Examples of ordered categorical response include Democrat, Independent, Republican; Yes, Maybe, No; Always, Frequently, Often, Rarely, Never.
  \vskip 12pt
  \item Examples of unordered categorical response include Football, Basketball, Baseball, Hockey; Train, Bus, Automobile, Walk; White, Black, Hispanic, Asian, Other.
  \end{itemize}
}

\frame[t] {
  \frametitle{Ordered Categorical: multinomial logit regression}
  \begin{itemize}
  \item Outcome $y$ can take on values $1,2,\ldots,K$, then the multinomial logit regression is given by
  \begin{eqnarray*}
  P(y>1)&=&\text{logit}^{-1}(X\beta)\\
  P(y>2)&=&\text{logit}^{-1}(X\beta-c_2)\\
  P(y>3)&=&\text{logit}^{-1}(X\beta-c_3)\\
  \ldots\\
  P(y>K-1)&=&\text{logit}^{-1}(X\beta-c_{K-1})\\
  \end{eqnarray*} 
  \item Equivalently, the model can be given by
  \[P(y=k)=P(y>k-1)-P(y>k)=\text{logit}^{-1}(X\beta-c_{k-1})-\text{logit}^{-1}(X\beta-c_k)\]
  \end{itemize}
}


\frame[t]{
  \frametitle{Latent Variable Formulation}
  \begin{itemize}
  \item
    This model might be easier to understand if we choose the latent variable formulation point of view.
    \begin{eqnarray*}
      y_i &=& \begin{cases} 1 & \text{if } z_i<0\\ 2 & \text{if } z_i\in(0,c_2) \\
      3 & \text{if } z_i\in(c_2,c_3)\\
       \ldots \\
      k-1 & \text{if } z_i\in(c_{k-1}, c_k)\\
      k & \text{if } z_i>c_k\end{cases}\\
      z_i &=& x_i^T\beta+\xi_i \quad \xi_i \text{ i.i.d. logistic} 
    \end{eqnarray*}

  \item
   With different error distribution, we can have multinomial probit and multinomial robit models with latent variable formulation.
  \end{itemize}
}

\frame[t]{
  \frametitle{\texttt{R} Implementation}
  \begin{itemize}
  \item \texttt{polr} (\textsl{proportional odds logistic regression}) in package \texttt{MASS} is used to fit this class of models.
  \vskip 12pt
  \item logti\ \texttt{polr(y $\sim$ x\_1 + x\_2, method=c('logit'))}
  \vskip 12pt
  \item probit\ \texttt{polr(y $\sim$ x\_1 + x\_2, method=c('probit'))}
  \vskip 12pt
  \item cauchit\ \texttt{polr(y $\sim$ x\_1 + x\_2, method=c('cauchit'))}
  \end{itemize}
}

\frame[t]{
  \frametitle{Unordered Categorical Response}
  \begin{itemize}
  \item When reponse is unordered categorical, to model is need more effort.
  \vskip 12pt
  \item Assume there are $J$ response categories, then the $i$th observation $y_i$ can be converted into $J$ binary reponse variables
  \begin{equation*}
  y_{ij}=\begin{cases} 1 & y_i\text{ is in category } j \\ 0 & \text{ otherwise}\end{cases}
  \end{equation*}
  \vskip 12pt
  \item Recall that in logistic regression, we have 
  \[\log\Big(\frac{p_i}{1-p_i}\Big)=X_i^T\beta\]
  \end{itemize}
}

\frame[t]{
  \frametitle{Unordered Categorical Response}
  \begin{itemize}
  \item In the unordered categorical response case, there are $\frac{J(J-1)}{2}$ pairs of categories.m 
  \begin{eqnarray*}
  \log\Big(\frac{p_{i1}}{p_{i2}}\Big)&=&X_i^T\beta_{12}\\
  \log\Big(\frac{p_{i1}}{p_{i3}}\Big)&=&X_i^T\beta_{13}\\
  \ldots
  \end{eqnarray*}
  \item Choose one category as baseline category (generally the last category $J$), the model is given by
  \[\log\Big(\frac{p_{ij}}{p_{iJ}}\Big)=X_i^T\beta_j, j=1,2,\ldots, J-1\]
  \end{itemize}
}

\frame[t]{
  \frametitle{Unordered Categorical Response}
  \begin{itemize}
  \item Equivalently, the model can be expressed as 
  \[\pi_{ij}=\frac{\exp(X_i^T\beta_j)}{1+\sum_{k=1}^{J-1}\exp(X_i^T\beta_k)}, j=1,2,\ldots, J-1\]
  \vskip 12pt
  \item Unordered categorical response is very unstructured, so there are a lot of parameters.
  \vskip 12pt
  \item We can use \texttt{mlogit} in package \texttt{mlogit} to fit the model\\
   \texttt{mlogit(y $\sim$ x\_1 + x\_2, reflevel='1')}
  \end{itemize}
}

\frame[t]{
  \frametitle{Generalized Linear Models}
  \begin{itemize}
  \item Generalized Linear Model is a way of unifying various kinds of regression models, including Linear Regression, Logistic Regression and Poisson Regression.
  \vskip 12pt
  \item It is DIFFERENT from General Linear Model.
  \vskip 12pt 
  \item There is also an unified way to do maximum likelihood estimation for GLM, which is called {\color{red} Iteratively Reweighted Least Squares} algorithm.
  \end{itemize}
}


\frame[t]{
  \frametitle{Generalized Linear Models}
  There are three components for GLM
  \begin{itemize}
  \item A probability distribution for response $y$, hopefully from the exponential family
    \vskip 24pt
  \item Linear predictor $\eta=X\beta$
    \vskip 24pt
  \item A link function $g$ such that $Ey=g^{-1}(X\beta)$
    \vskip 24pt
  \end{itemize}
}


\end{document}
