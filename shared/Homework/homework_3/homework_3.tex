% Essential Formatting

\documentclass[12pt]{article}
\usepackage{epsfig,amsmath,amsthm,amssymb}
\usepackage[questions, answersheet]{urmathtest}[2001/05/12]
%\usepackage[answersheet]{urmathtest}[2001/05/12]
%\usepackage[answers]{urmathtest}[2001/05/12]


% For use with pdflatex
% \pdfpagewidth\paperwidth
% \pdfpageheight\paperheight

% Basic User Defs

\def\ds{\displaystyle}

\newcommand{\ansbox}[1]
{\work{
  \pos\hfill \framebox[#1][l]{ANSWER:\rule[-.3in]{0in}{.7in}}
}{}}

\newcommand{\ansrectangle}
{\work{
  \pos\hfill \framebox[6in][l]{ANSWER:\rule[-.3in]{0in}{.7in}}
}{}}


% Beginning of the Document

\begin{document}
\examtitle{LINEAR REGRESSION MODELS W4315}{HOMEWORK 3}{09/16/2009}
 \begin{center}
  Instructor: Frank Wood
 \end{center}
%%\studentinfo
\instructions{
  %\textbf{Circle your Instructor's Name along with the Lecture Time:}



  \begin{itemize}
  \item
    \textbf{Please show all your work.
            You may use back pages if necessary.}
  %\item
   % \textbf{Please put your \underline{simplified}
   %         final answers in the spaces provided.}
  \end{itemize}
}
\finishfirstpage

% Problems Start Here % ----------------------------------------------------- %


\problem{50} {\footnote[1]{This is
problem 2.5 in "Applied Linear Regression Models(4th edition)" by
Kutner etc.} 
 Refer to \textbf{Copier maintenance} Problem 1.20.\\
 a. Estimate the change in the mean service time when the number of copiers serviced increases by one. Use a 90 percent confidence interval. Interpret your confidence interval.\\
 b. Conduct a \textit{t} test to determine whether or not there is a linear association between X and Y here; control the $\alpha$ risk at .10. State the alternatives, decision rule, and conclusion. What is the \textit{P}-value of your test?\\
 c. Are your results in parts (a) and (b) consistent? Explain.\\
 d. The manufacturer has suggested that the mean required time should not increase by more than 14 minutes for each additional copier that is serviced on a service call. Conduct a test to decide whether this standard is being satisfied by Tri-City. Control the risk of a Type I error at .05. State the alternatives, decision rule, and conclusion. What is the \textit{P}-value of the test?\\
 e. Does $b_0$ give any relevant information here about the ``start-up'' time on calls-i.e., about the time required before service work is begun on the copiers at a customer location? 
   }
 { \vfill
  \answer
} { }

\problem{20} {\footnote[2]{This is
problem 2.19 in "Applied Linear Regression Models(4th edition)" by
Kutner etc.} 
 Consider the test problem in a normal error regression model:
 \begin{center}
 $Y_i=\beta_0+\beta_1 X_i+\epsilon_i$\\
 \end{center}
 where:\\
 \begin{center}
 $\beta_0$ and $\beta_1$ are parameters\\
 $X_i$ are known constants\\
 $\epsilon_i$ are independent $N(0,\sigma^2)$
 \end{center}
 When testing whether or not $\beta_1=0$, why is the \textit{F} test a one-sided test even though $H_a$ includes both $\beta_1<0$ and $\beta_1>0$? [Hint: refer to the following problem]
   }
 { \vfill
  \answer
} { }

\problem{30} {\footnote[3]{This is
problem 2.57 in "Applied Linear Regression Models(4th edition)" by
Kutner etc.} 
 Consider the same normal regression model as in problem 2.\\
 a. When testing $H_0:~\beta_1=5$ versus $H_a:~\beta_1\ne 5$ by means of a general linear test, what is the reduced model? What are the degrees of freedom $df_R$?\\
 b. When testing $H_0:~ \beta_0=2,~\beta_1=5$ versus $H_a:$~not both $\beta_0=2$ and $\beta_1=5$ by means of a general linear test, what is the reduced model? What are the degrees of freedom $df_R$? 
}
 { \vfill
  \answer
} {}



% Problems End Here % ------------------------------------------------------- %

\problemsdone
\end{document}

% End of the Document
