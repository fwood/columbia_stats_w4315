% Essential Formatting

\documentclass[12pt]{article}
\usepackage{epsfig,amsmath,amsthm,amssymb}
\usepackage[questions, answersheet]{urmathtest}[2001/05/12]
%\usepackage[answersheet]{urmathtest}[2001/05/12]
%\usepackage[answers]{urmathtest}[2001/05/12]


% For use with pdflatex
% \pdfpagewidth\paperwidth
% \pdfpageheight\paperheight

% Basic User Defs

\def\ds{\displaystyle}

\newcommand{\ansbox}[1]
{\work{
  \pos\hfill \framebox[#1][l]{ANSWER:\rule[-.3in]{0in}{.7in}}
}{}}

\newcommand{\ansrectangle}
{\work{
  \pos\hfill \framebox[6in][l]{ANSWER:\rule[-.3in]{0in}{.7in}}
}{}}


% Beginning of the Document

\begin{document}
\examtitle{LINEAR REGRESSION MODELS W4315}{HOMEWORK 1}{09/16/2009}
 \begin{center}
  Professor: Frank Wood
 \end{center}
%%\studentinfo
\instructions{
  %\textbf{Circle your Instructor's Name along with the Lecture Time:}



  \begin{itemize}
  \item
    \textbf{Please show all your work.
            You may use back pages if necessary.}
  %\item
   % \textbf{Please put your \underline{simplified}
   %         final answers in the spaces provided.}
  \end{itemize}
}
\finishfirstpage

% Problems Start Here % ----------------------------------------------------- %


\problem{20} { Let $Y_i = \beta_0 + \beta_1 X_i + \epsilon_i$ be a
linear regression model with distribution of error terms unspecified
(but with mean $E(\epsilon) = 0$ and variance $V(\epsilon_i) =
\sigma^2$ ($\sigma^2$ finite) known).  Show that $s^2 = MSE =
\frac{\sum(Y_i-\hat Y_i)^2}{n-2}$ is an unbiased estimator for
$\sigma^2$.  $\hat Y_i = b_0 + b_1 X_i$ where $b_0 = \bar Y - b_1
\bar X$ and $b_1 = \frac{\sum_i((X_i-\bar X)(Y_i - \bar
Y))}{\sum_i(X_i-\bar X)}$ } { \vfill
  \answer
}
{
First, let's denote the followings:\\
$\hat{e_i}=y_i-\hat{y_i}$\\
$SXX=\displaystyle\sum_{i=1}^n (x_i-\bar{x})^2$\\
$SYY=\displaystyle\sum_{i=1}^n (y_i-\bar{y})^2$\\
$SXY=\displaystyle\sum_{i=1}^n (x_i-\bar{x})(y_i-\bar{y})$\\
\\
Now we set out to prove the following equation which accomplishes
essentially the final result:\\
$Var\hat{e_i}=E\hat{e_i}^2=(\frac{n-2}{n}+\frac{1}{SXX}(\frac{1}{n}\displaystyle\sum_{j=1}^nx_j^2+x_i^2-2(x_i-\bar{x})^2-2x_i\bar{x}))\sigma^2$\\
To prove the above, realize that:\\
\begin{align*}
Var(\hat{e_i})&=Var(y_i-\hat{\beta_0}-\hat{\beta_1}x_i)\\
              &=Var((y_i-\beta_0-\beta_1x_i)-(\hat{\beta_0}-\beta_0)-x_i(\hat{\beta_1}-\beta_1))\\
              &=Var(y_i)+Var(\hat{\beta_0})+x_i^2Var(\hat{\beta_1})-2Cov(y_i,\hat{\beta_0})-2x_iCov(y_i,\hat{\beta_1})+2x_iCov(\hat{\beta_0},\hat{\beta_1})\\
\end{align*}
The last equation holds because the covariance between any random
variable and a constant is zero, and all the $y_i$'s are independent
entailing that the $Cov(y_i,y_j)=0,i\not=j$\\
$Var(y_i)=\sigma^2$\\
Notice that(some algebras needed here, and the following tricks are
crucial in reducing the amount of calculation):\\
$\sum{(x_i-\bar{x})}=0$\\
$\beta_1=\frac{\sum{x_i-\bar{x}}y_i}{SXX}$\\
So now we have:\\
\begin{align*}
 Var(\beta_1)&=Var(\frac{SXY}{SXX})\\
             &=Var(\frac{\sum{(x_i-\bar{x})}y_i}{SXX})\\
             &=\frac{1}{SXX^2}\sum{x_i-\bar{x}}^2Var(y_i)\\
             &=\frac{\sigma^2}{SXX}\\
\end{align*}
And:\\
\begin{align*}
 Var(\beta_0)&=Var(\bar{y}-\hat{\beta_1}\bar{x})\\
             &=Var(\sum{(\frac{1}{n}-\frac{(x_i-\bar{x})\bar{x}}{SXX})y_i})\\
             &=\sum{(\frac{1}{n}-\frac{x_i-\bar{x}}{SXX}\bar{x})^2}\sigma^2\\
             &=\sum{[\frac{1}{n^2}+\frac{SXX*\bar{x}^2}{SXX^2}-\frac{2}{n}\frac{\bar{x}(x_i-\bar{x})}{XSS}]}\sigma^2\\
             &=[\frac{1}{n}+\frac{n\bar{x}^2}{SXX}]\sigma^2\\
             &=\frac{\sum{x_i}^2}{n*SXX}\sigma^2
\end{align*}
For the other terms in the decomposition of $Var(\hat{e_i})$, we
have:\\
\begin{align*}
 Cov(y_i,\hat{\beta_1})&=Cov(y_i,\frac{sum{x_i-\bar{x}}y_i}{SXX})\\
                       &=\frac{x_i-\bar{x}}{SXX}Var(y_i)\\
                       &=\frac{x_i-\bar{x}}{SXX}\sigma^2
\end{align*}
and:\\
\begin{align*}
 Cov(y_i,\hat{\beta_0})&=Cov(y_i,\bar{y}-\hat{\beta_1}\bar{x})\\
                       &=Cov(y_i,\frac{\sum{y_i}}{n}-\sum{(x_i-\bar{x})y_i}{SXX}\bar{x})\\
                       &=\frac{\sigma^2}{n}+\bar{x}\frac{x_i-\bar{x}}{SXX}\sigma^2
\end{align*}
At last, we have:\\
\begin{align*}
 Cov(\hat{\beta_0},\hat{\beta_1})&=Cov(\bar{y}-\hat{\beta_1}\bar{x},\hat{\beta_1})\\
                                 &=Cov(\frac{\sum{y_i}}{n}-\sum{\frac{(x_i-\bar{x})\bar{x}}{SXX}}y_i,\sum{\frac{(x_i-\bar{x})y_i}{SXX}})\\
                                 &=\displaystyle\sum_{i=i}^{n}(\frac{1}{n}-\frac{x_i-\bar{x}}{SXX}\bar{x})\frac{x_i-\bar{x}}{SXX}\sigma^2\\
                                 &=-\frac{\bar{x}}{SXX}\sigma^2
\end{align*}
Then plug in all the parts back to the decomposition of
$Var(\hat{e_i})$, we have:\\
$Var(\hat{e_i})=(\frac{n-1}{n}+\frac{1}{SXX}(\frac{1}{n}\displaystyle\sum_{j=1}^{n}{x_j}^2+x_i^2-2(x_i-\bar{x})^2-2x_i\bar{x}))\sigma^2$\\
\noindent Thus,\\
\begin{align*}
 E\hat{\sigma}^2&=\frac{1}{n}\displaystyle\sum_{i=1}^{n}E\hat{e_i}^2\\
                &=\frac{1}{n}\displaystyle\sum_{i=1}^{n}[\frac{n-2}{n}+\frac{1}{SXX}(\frac{1}{n}\displaystyle\sum_{j=1}^{n}x_j^2+x_i^2-2(x_i-\bar{x})^2-2x_i\bar{x})]\sigma^2\\
                &=[\frac{n-2}{n}+\frac{1}{nS_{xx}}\{\displaystyle\sum_{j=1}^{n}x_j^2+\displaystyle\sum_{i=1}^{n}x_i^2-2SXX-2\frac{1}{n}(\displaystyle\sum_{i=1}^{n}x_i)^2\}]\sigma^2\\
                &=(\frac{n-2}{n}+0)\sigma^2\\
                &=\frac{n-2}{n}\sigma^2\\
\end{align*}
where the third equation holds because:
$\sum{}{}x_i\bar{x}=\frac{1}{n}(\sum{}{} x_i)^2$\\
and the second to last equation holds since
$\sum{}{} x_i^2-\frac{1}{n}(\sum{}{} x_i)^2=SXX$\\
From the above equation, the result flows. \\
}

\problem{20} { Derive the maximimum likelihood estimators $\hat
\beta_0, \hat \beta_1,$ and $\hat \sigma^2$ for parameters $\beta_0,
\beta_1,$ and $\sigma^2$ for the normal linear regression model
(i.e.~$\epsilon_i \sim_{iid} N(0,\sigma^2)$). } { \vfill
  \answer
}
{
To figure the MLE of the parameters, we need to first write down
the likelihood function of the data, so under normal assumption, we
have the log-likelihood function as follows:\\
$logL(\beta_0,\beta_1,\sigma^2|x,y)=-\frac{n}{2}log(2\pi)-\frac{n}{2}log\sigma^2-\frac{\displaystyle\sum_{i=1}^{n}(y_i-\beta_0-\beta_1x_i)^2}{2\sigma^2}.$\\
For any fixed value of $\sigma^2$, $logL$ is maximized as a function
of $\beta_0$ and $\beta_1$, that minimize
\begin{eqnarray}
\displaystyle\sum_{i=1}^{n}(y_i-\beta_0-\beta_1x_i)^2
\end{eqnarray}
But to minimize this function is just to principle behind LSE, so
it's apparent that the MLE of $\beta_0$ and $\beta_1$ are the same
as their LSE's. Now, substituting in the log-likelihood, to find the
MLE of $\sigma^2$ we need to maximize\\
\[-\frac{n}{2}log(2\pi)-\frac{n}{2}log\sigma^2-\frac{\displaystyle\sum_{i=1}^{n}(y_i-\hat{\beta_0}-\hat{\beta_1}x_i)^2}{2\sigma^2}\]\\
This maximization problem is nothing but MLE of $\sigma^2$ in
ordinary normal sampling problems, which is easily given as\\
\[\hat{\sigma}^2=\frac{1}{n}\displaystyle\sum_{i=1}^{n}(y_i-\hat{\beta_0}-\hat{\beta_1}x_i)^2\]\\
If you are not familiar with the MLE in normal sampling setting, you
can take derivative with respect to $\sigma^2$ (N.B. not $\sigma$),
and then set the derivative to be zero. The solution of the equation
is just the MLE of $\sigma^2$.\\
}

\problem{10} { File `unif.txt' contains 200 numbers randomly
generated from a uniform distribution U(2,5). Read these numbers into MATLAB (using `textread' for instance) and do the following: \\
a. Take these 200 number as a population. Use command 'randsample'
of MATLAB, draw 100 numbers out of this population randomly with
replacement and plot one histogram of
these 100 numbers.\\
b. Use command 'rand' in MATLAB, draw 100 samples directly from
U(2,5) and plot another histogram of these numbers.\\
c. Compare the two histograms, what can you say about the difference
between the distribution of samples from the population and from the
uniform distribution itself? For the convenience of comparison, you
may want to overlay two histograms onto one graph and to see if any
apparent difference. }
 { \vfill
  \answer
} { }

\problem{10} { File `normal.txt' contains 200
randomly
generated numbers from Normal distribution N(-1,$2^2$). Like in problem 3, do the following: \\
a. Take these 200 number as a population. Use command `randsample'
of MATLAB, draw 100 numbers out of this population randomly with
replacement and plot a histogram of
these 100 numbers.\\
b. Use command `randn' in MATLAB, draw 100 samples directly from
N(-1,$2^2$) and plot a histogram of these numbers.\\
c. Compare the two histograms, can you get the similar conclusion as
that of problem 3?}
 { \vfill
  \answer
} { }



\problem{40} { \textbf{Copier maintenance.}\footnote[1]{This is
problem 1.20 in ``Applied Linear Regression Models(4th edition)'' by
Kutner etc. )} The Tri-City Office Equipment Corporation sells an
imported copier on a franchise basis and performs preventive
maintenance and repair services on this copier. The data below have
been collected from 45 recent calls on users to perform routine
preventive number of minutes spent by the service person. Assume
that first-order regression model($Y_i=b_0+b_1 X_i+\epsilon_i$) is
appropriate.
\begin{table}[htdp]
\begin{center}
\begin{tabular}{rcrcrcrcrcrcrcrc}
\textbf{i:} &\textbf{1} &\textbf{2}& \textbf{3}&...&\textbf{43}&\textbf{44}&\textbf{45}\\
\hline \textbf{$X_i$} &2 &4 &3 &... &2 &4 &5\\
\textbf{$Y_i$} &20 &60 &46 &... &27 &61 &77
\end{tabular}
\end{center}
\end{table}\\

a. Obtain estimated regression function.\\
b. Plot the estimated regression function and the data. How well
does the estimated regression function fit the data?\\
c. interpret $b_0$ in your estimated regression function. Does $b_0$
provide any relevant information here? Explain.\\
d. Obtain a point estimate of the mean service time when $X=5$
copiers are serviced.\\
\indent~~~~Notice: You can get data for this problem on
www.mhhe.com/KutnerALRM4e. Use MATLAB, do not
use any other programming language. Only basic MATLAB operators are
allowed, do not use any built-in functions to do the regression, i.e.~the function ``regress'' cannot be used except, perhaps, to verify that your answer is correct before submitting your own implementation. \\}
 { \vfill
  \answer
} { }


% Problems End Here % ------------------------------------------------------- %

\problemsdone
\end{document}

% End of the Document
