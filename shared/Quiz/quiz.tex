% Essential Formatting

\documentclass[12pt]{article}
\usepackage{epsfig,amsmath,amsthm,amssymb}
\usepackage[questions]{urmathtest}[2001/05/12]
%\usepackage[answersheet]{urmathtest}[2001/05/12]
%\usepackage[answers]{urmathtest}[2001/05/12]

% For use with pdflatex
% \pdfpagewidth\paperwidth
% \pdfpageheight\paperheight

% Basic User Defs

\def\ds{\displaystyle}

\newcommand{\ansbox}[1]
{\work{
  \pos\hfill \framebox[#1][l]{ANSWER:\rule[-.3in]{0in}{.7in}}
}{}}

\newcommand{\ansrectangle}
{\work{
  \pos\hfill \framebox[6in][l]{ANSWER:\rule[-.3in]{0in}{.7in}}
}{}}

% Beginning of the Document

\begin{document}
\examtitle{LINEAR REGRESSION MODELS W4315}{Quiz}{}
 \begin{center}
  Instructor: Frank Wood 
 \end{center}
\studentinfo
\instructions{
  %\textbf{Circle your Instructor's Name along with the Lecture Time:}



  \begin{itemize}
  \item
    \textbf{Please show all your work.
            You may use back pages if necessary.}
  %\item
   % \textbf{Please put your \underline{simplified}
   %         final answers in the spaces provided.}
  \end{itemize}
}
\finishfirstpage

% Problems Start Here % ----------------------------------------------------- %


\problem{30}
{
Prove the Law of Total Probability. Suppose events $A_1, A_2, \cdots, A_n$ form a \emph{partition} of $\Omega$. That is, the events are mutually disjoint and their union is all of $\Omega$. Then for any other event $B$, we have
$$P(B)=P(A_1)\times P(B|A_1)+P(A_2)\times P(B|A_2)+\cdots+P(A_n)\times P(B|A_n).$$
}
{

\vfill
 \newpage

}
{
}

%\problem{30}
%{
%\abcs
%\item Suppose there are events $A_1, A_2, \cdots, A_n$ in $\Omega$. For any other event $B$, please indicate under %which conditions that $A_1, A_2, \cdots, A_n$ satisfy, the following equation holds:
%    \begin{equation}
%    P(B)=P(A_1)\times P(B|A_1)+P(A_2)\times P(B|A_2)+\cdots+P(A_n)\times P(B|A_n).
%    \end{equation}
%\item Prove (1) under the conditions you just stated.
%\endabcs
%}
%{
%
%\vfill
% \newpage
%
%}
%{
%}
%
%\problem{40}
%{
%Please solve the following equations:
%\[
%\left\{
%\begin{array}{ccc}
% x_1 + 2x_2 + 3x_3 &= 1  \\
% 3x_1 + 5x_2 - x_3 &= 3  \\
% x_1 + x_2 - 7x_3 &= 1
%\end{array}
%\right.
%\]
%}
%{
%
%\vfill
% \newpage
%
%}
%{
%}

\problem{40}
{
\[
\left\{
\begin{array}{ccc}
 x_1 + 2x_2 + 3x_3 &= \alpha  \\
 3x_1 + 5x_2 - x_3 &= \beta \\
 x_1 + x_2 - 7x_3 &= \gamma
\end{array}
\right.
\]
Under what conditions of $\alpha,\beta,\gamma$, the above equations have (a)unique solution (b)no solution (c)infinite solutions.
}
{

\vfill
 \newpage

}
{
}

%\problem{30}
%{
%Suppose $X_i\sim N(\mu, \sigma_i^2)$, $i=1, 2, \cdots, n$ where $\sigma_i^2$'s are known. Derive the Maximum %Likelihood Estimator of $\mu$.
%}
%{
%
%\vfill
% \newpage
%
%}
%{
%}

\problem{30}
{
Suppose $$Y_i = X_i\beta + \epsilon_i,$$ where $\epsilon_i \stackrel{iid}{\sim} N(0, \sigma_1^2)$, $i=1, 2, \cdots, n$ and $\epsilon_i \stackrel{iid}{\sim} N(0, \sigma_2^2)$, $i=n+1, n+2, \cdots, n+m$. Derive the Maximum Likelihood Estimator of $\beta$, $\sigma_1^2$ and $\sigma_2^2$.
}
{

\vfill
 \newpage

}
{
}
% Problems End Here % ------------------------------------------------------- %

\problemsdone
\end{document}

% End of the Document
